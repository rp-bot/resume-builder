\documentclass[letterpaper,11pt]{article}

\usepackage{latexsym}
\usepackage[empty]{fullpage}
\usepackage{titlesec}
\usepackage{marvosym}
\usepackage[usenames,dvipsnames]{color}
\usepackage{verbatim}
\usepackage{enumitem}
\usepackage{hyperref}
\usepackage{fancyhdr}
\usepackage[english]{babel}
\usepackage{tabularx}
\usepackage{fontawesome}
\usepackage{changepage}
\usepackage[margin=0.2in]{geometry} % Adjust margins here
\input{glyphtounicode}

\definecolor{linkcolor}{RGB}{0, 70, 150}

\hypersetup{colorlinks = true}

\pagestyle{fancy}
\fancyhf{} % clear all header and footer fields
\fancyfoot{}
\renewcommand{\headrulewidth}{0pt}
\renewcommand{\footrulewidth}{0pt}
\setlength{\footskip}{4.08003pt}
% Adjust margins
% \addtolength{\oddsidemargin}{-0.5in}
% \addtolength{\evensidemargin}{-0.5in}
% \addtolength{\textwidth}{1in}
% \addtolength{\topmargin}{-.5in}
% \addtolength{\textheight}{1.0in}
\setlength{\tabcolsep}{0in}
\urlstyle{same}

\raggedbottom
\raggedright
\setlength{\tabcolsep}{0in}

% Sections formatting
\titleformat{\section}{
  \vspace{-6pt}\scshape\raggedright\large
}{}{0em}{}[\color{black}\titlerule \vspace{-5pt}]

% Ensure that generate pdf is machine readable/ATS parsable
\pdfgentounicode=1

% Custom commands

\newcommand{\resumeItem}[1]{
  \item\small{
    {#1 \vspace{-2pt}}
  }
}
\newcommand{\resumeSubheading}[4]{
  \item
    \begin{tabular*}{0.97\textwidth}[t]{l@{\extracolsep{\fill}}r}
      \normalsize \textbf{#1} & \scriptsize#2 \\
      \small #3  \\
    \end{tabular*}\vspace{-12pt}
}
\newcommand{\resumeEducationHeading}[4]{
  \item
    \begin{tabular*}{0.97\textwidth}[t]{l@{\extracolsep{\fill}}r}
      \textbf{#1} & \scriptsize #2 \\
      \small#3 & \\
    \end{tabular*}
    \vspace{-3pt} % Adjust this value to control space between third and fourth arguments
    % \begin{tabular*}{0.97\textwidth}[t]{@{}l@{}}
    %   \parbox[t]{0.8\textwidth}{\small #4} \\
    % \end{tabular*}
}
\newcommand{\resumeProjectHeading}[2]{
    \vspace{3pt}\item
    \begin{tabular*}{0.97\textwidth}{l@{\extracolsep{\fill}}r}
      \small#1 & #2 \\
    \end{tabular*}\vspace{-9pt}
}

\renewcommand\labelitemii{$\vcenter{\hbox{\tiny$\bullet$}}$}

\newcommand{\resumeSubHeadingListStart}{\begin{itemize}[leftmargin=0.15in, label={}]}
\newcommand{\resumeSubHeadingListEnd}{\end{itemize}}

\newcommand{\resumeItemListStart}{\begin{itemize}\vspace{-8pt}}
\newcommand{\resumeItemListEnd}{\end{itemize}\vspace{-5pt}}

\newcommand{\resumeSectionBody}[1]{
  \vspace{0.7em}
  \small #1
  \vspace{-5pt}
}


\begin{document}

%---------- HEADING ----------

\hypersetup{urlcolor = black} % Don't color link text in the header.

\begin{center}
    \textbf{\LARGE __NAME__} \\ \vspace{3pt}
    \small
    % \faMobile \hspace{.5pt} \href{tel:6127479175}{612-747-9175}
    % $|$
     \hspace{.5pt} {\underline{__EMAIL__}}
    $|$
     \hspace{.5pt} {\underline{LinkedIn: __LINKEDIN__}}
    $|$
     \hspace{.5pt} {\underline{GitHub: __GITHUB__}}
    $|$
 \hspace{.5pt} {\underline{__WEBSITE__}}
\end{center}
\vspace{-18pt}

\hypersetup{urlcolor = linkcolor} % Change back to link color for the rest of the document.

\section{Professional Summary}
__SUMMARY__


\section{Technical Expertise}
\resumeSubHeadingListStart
  \resumeItem{
    \small{
      __SKILLS__\\[-5pt]
    }
  }
\resumeSubHeadingListEnd
\vspace{-5pt}

% \section{Professional Experience}
%   \resumeSubHeadingListStart
        
%     \resumeSubheading
%       {Music Informatics Lab, Georgia Tech}{Atlanta, GA $|$ Jan 2025 – Present}{}
      
%       \resumeItemListStart
%         \resumeItem{Contributed to AI-driven research on music analysis and generation, focusing on Music Information Retrieval.}
%         % \resumeItem{Developed machine learning models for extracting and interpreting musical features from drum recordings, improving analysis accuracy by 15\%}
%         \resumeItem{\textbf{Implemented} \textbf{CUDA}-based \textbf{GPU acceleration} and mem-mapping, achieving 20x speedup for large audio dataset processing.}
%         \resumeItem{Establish research methodologies including literature reviews and baseline model development.}
%         % \resumeItem{Worked under the guidance of Dr. Alexander Lerch, Associate Dean of Research and Creative Practice.}

%       \resumeItemListEnd
%   \resumeSubHeadingListEnd
%   \resumeSubHeadingListStart
        
%     \resumeSubheading
%       {Freelance Full Stack Developer}{Atlanta, GA $|$ October 2023 – Present}{}
      
%       \resumeItemListStart
%         \resumeItem{\textbf{Engineered} scalable real-time apps for 10k+ concurrent users with low-latency media streaming.}
%         \resumeItem{\textbf{Reduced} application latency and load times to achieve near-instantaneous user interaction.}
%         \resumeItem{\textbf{Implemented} secure authentication systems (Auth0, Supabase/Firebase) to protect user data and assets.}
%         \resumeItem{Developed high-availability solutions for data-intensive applications, enhancing platform reliability.}
%       \resumeItemListEnd
%   \resumeSubHeadingListEnd
%   \resumeSubHeadingListStart
        
%     \resumeSubheading
%       {Center for Research and Learning(CRL)}{Indianapolis, IN $|$ May - Aug 2023}{}
      
%       \resumeItemListStart
%         \resumeItem{\textbf{Engineered} neural network using PyTorch implementing \textbf{Transformer Architecture} for music generation.}
%         \resumeItem{\textbf{Optimized} model to generate chord progressions with 60\% more diverse velocity, enhancing human-like sound quality.}
%         \resumeItem{Presented research findings at CRL Symposium 2023, receiving academic scholarship for outstanding performance.}
%         % \resumeItem{Presented technical findings at CRL Symposium 2023, contributing to Music Information Retrieval research.}
%         % \resumeItem{Received academic scholarship for outstanding research performance during the semester.}
%         % \resumeItem{Collaborated with Dr. Jason Palamara, Professor of Music Technology at Indiana University}
%       \resumeItemListEnd
%   \resumeSubHeadingListEnd
% \vspace{-15pt}

% \section{Projects}
%     \resumeSubHeadingListStart
%     \resumeProjectHeading
%       {{\textbf{\underline{MIDI Gen AI}}} $|$ Generative AI}{}
      
%       \resumeSectionBody{
%          Designed and \textbf{implemented} a \textbf{Large Language Model (LLM)} that predicts musical chords, trained on 20.9M MIDI tokens.\\
%          Applied advanced music theory principles to enhance generation quality and musical coherence.
%       }
% % 

%       %  \resumeProjectHeading
%       % {\href{https://github.com/rp-bot/audio-sentiment-analysis}{\textbf{Audio Sentiment Analysis}} $|$ Python/PyTorch/CNN}{}

%       % \resumeSectionBody{
%       % {A CNN audio sentiment analysis model for movie soundtracks. Producing an accuracy of 92\%.}
%       % }

%       % \resumeProjectHeading
%       % {\href{https://github.com/L42i/SPRAWL}{\textbf{SPRAWL: Communication and Sound-Reactive Light Node}} $|$  SuperCollider/C/C++/Shell}{}

%       % \resumeSectionBody{
%       % {Contributed to a research project in the L42i (Immersion) Lab at Georgia Tech. Collaborated with the team to enhance project communication and developed a sound-reactive light node.}
%       % }
% \resumeProjectHeading
%       {{\textbf{\underline{Eco-charge}}} $|$  Hackathon 3rd Place Winner}{}

%       \resumeSectionBody{
%       Developed a tool optimizing EV charging schedules to minimize \textbf{CO$_2$ emissions} during the vehicle use phase.
%       Applied data analytics to predict optimal charging windows based on grid carbon intensity
%       }
% % https://github.com/rp-bot/computer-vision-algorithms
      
%       % \resumeProjectHeading
%       % {\href{https://github.com/rp-bot/DSP-Fundamentals}{\textbf{DSP Fundamentals}} $|$ Python/numpy}{}

%       % \resumeSectionBody{
%       % {The project simplifies DSP learning by enabling direct practical engagement without complex setup requirements.}
%       % }
% % \resumeProjectHeading
% %       {\href{https://github.com/rp-bot/ConvolutionReverb}{\textbf{Convolution Reverb Plugin}} $|$ C/C++/JUCE/DSP}{}

% %       \resumeSectionBody{
% %       {A convolution reverb plugin with JUCE, optimizing for realistic sound and low-latency performance.}
% %       }

%  % \resumeProjectHeading
%  %      {\href{https://github.com/rp-bot/daw-os-kernel}{\textbf{DAW-OS Kernel}} $|$ Assembly/C}{}

%  %      \resumeSectionBody{
%  %      {Designing a custom operating system optimized for Digital Audio Workstations (DAWs), focused on providing a robust and high-performance environment for seamless audio production.}
%  %      }
%       \resumeProjectHeading
%       {{\textbf{\underline{AI-Powered Operations Platform}}} $|$ Freelance project}{}

%     \resumeSectionBody{
%     Architected and engineered a comprehensive, cloud-native platform to automate complex workflows and streamline core operational processes for an enterprise client.
%     }
%       \resumeProjectHeading
%       {{\textbf{\underline{Home Security System}}} $|$ IoT Solution}{}

%       \resumeSectionBody{
%         Engineered \textbf{Arduino and ESP32 Wi-Fi} system for real-time security monitoring with sensor integration.
%         Implemented custom \textbf{firmware} for reliable data streaming and alert mechanisms.
%       }
%       %  \resumeProjectHeading
%       % {{\textbf{\underline{Vision Synth: Hand Gesture Music Interface}}} $|$ \textbf{Computer Vision} Music Interface}{}

%       % \resumeSectionBody{
%       % Created gesture-based music generation system using YOLO hand detection and neural network processing.\\
%       % Implemented real-time tracking via webcam with low-latency audio response for intuitive musical expression.
%       % }

%  % \resumeProjectHeading
%  %      {\href{https://github.com/rp-bot/computer-vision-algorithms}{\textbf{Computer Vision Algorithms}} $|$ C++}{}

%  %      \resumeSectionBody{
%  %      Implemented fundamental computer vision algorithms, including Sobel operators, convolution, and edge detection
%  %      }

     
%     \resumeSubHeadingListEnd
\section{Education}
  % \vspace{1pt}
  \resumeSubHeadingListStart
    __EDUCATION_LIST_ITEM__
    
      
    % \resumeEducationHeading
    %   {__SCHOOL__}{__LOCATION__ $|$ __DATE__}
    %   {__DEGREE__  \footnotesize{}}{\textbf{Relevant Coursework:} __COURSEWORK__}
    % \resumeEducationHeading
    %   {__SCHOOL__}{__LOCATION__ $|$ __DATE__}
    %   {__DEGREE__  \footnotesize{}}{\textbf{Relevant Coursework:} __COURSEWORK__}
\resumeSubHeadingListEnd




\end{document}